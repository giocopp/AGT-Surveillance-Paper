% main.tex - basic LaTeX template
\documentclass[11pt]{article}

% Encoding and fonts
\usepackage[T1]{fontenc}
\usepackage[utf8]{inputenc}
\usepackage{lmodern}
\usepackage{microtype}
\usepackage{graphicx} % Required for inserting images

% references 
\usepackage[style=apa, backend=biber]{biblatex}

\addbibresource{references.bib}
% Math
\usepackage{amsmath,amssymb,amsthm}

% Hyperlinks
\usepackage[colorlinks=true,linkcolor=blue,citecolor=blue,urlcolor=blue]{hyperref}

% Title data
\title{Title goes here}
\author{Giulia Maria Petrilli, Giorgio Coppola}
\date{\today}

\begin{document}

\maketitle 


\begin{abstract}
A short abstract.
\end{abstract}

\section{Introduction}
The literature on digital authoritarianism shows growing concerns about the use of digital technologies by states to monitor, surveil and control their population (\cite{Dragu_Lupu_2021}).
In this context, Amnesty International has reported that Israeli authorities have been using a facial recognition system known as Red Wolf to track Palestinians, and automate restrictions on their freedom of movement (\citeyear{Amnesty_2023_FacialRecognition}).
Game theory models have been used to study the strategic interactions between authoritarian regimes and citizens in the context of digital surveillance (\cite{Dragu_Lupu_2021};\cite{Bambauer_Zarsky_2018}).
This paper places itself within this literature, and aims to model the interaction between an authoritarian state and its citizens in the context of facial recognition technology used for surveillance and control.
RESEARCH QUESTION OPTIONS:
The guiding research question is: \textit{How is the tool currently being used in public policy and how does this infringe on Palestinians rights and freedoms?}
The guiding research question is: \textit{How is the Game theory concept model the interaction between an authoritarian state and its citizens in the context of facial recognition technology?}

\section{Tool}
This section describes the concept from the course 

\section{How It is being Used}
This section describes the situation in the Blue Wolf case and lists the elements from a game theory perspective.

\section{Implications for Public Policy}


% Bibliography (if needed)
% \bibliographystyle{plain}
% \bibliography{references}
\printbibliography
\end{document}